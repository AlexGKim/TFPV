\section{Inferring the Absolute Magnitude of an Individual Galaxy from its Rotation Velocity}
\label{sec:infer_single_galaxy_y}

Assume we have already fit the TF model to a (possibly selection-corrected) sample and obtained the posterior
\begin{equation}
p(\theta \mid \{\hat x_i,\hat y_i\}_{i=1}^N),
\qquad
\theta \equiv (s,c,\sigma_{\rm int,x},\sigma_{\rm int,y}).
\end{equation}
We now consider a \emph{new} galaxy, denoted by ``$\star$'', for which we measure a rotation-velocity proxy
$\hat x_\star$ with uncertainty $\sigma_{x,\star}$, and we wish to infer its true absolute magnitude $y_\star$
(or predict what absolute magnitude we should expect given $\hat x_\star$ and the TF calibration).

\subsection{Posterior predictive distribution}
The desired quantity is the posterior predictive distribution
\begin{equation}
p(y_\star \mid \hat x_\star, \sigma_{x,\star}, \{\hat x_i,\hat y_i\})
=
\int d\theta\;
p\!\left(y_\star \mid \hat x_\star, \sigma_{x,\star}, \theta\right)\,
p\!\left(\theta \mid \{\hat x_i,\hat y_i\}\right).
\label{eq:posterior_predictive_y}
\end{equation}
In practice, if we have posterior draws $\{\theta^{(m)}\}_{m=1}^M$, the integral is approximated by a mixture
\begin{equation}
p(y_\star \mid \hat x_\star, \cdots)\;\approx\; \frac{1}{M}\sum_{m=1}^M
p\!\left(y_\star \mid \hat x_\star, \sigma_{x,\star}, \theta^{(m)}\right).
\end{equation}

\subsection{Closed form when $p(y_{\rm TF})$ is (improper) uniform on $\mathbb{R}$}
Adopt the same generative structure as in \S\ref{sec:model}, and define
\begin{equation}
\sigma_{1,\star}^2 \equiv \sigma_{x,\star}^2 + \sigma_{\rm int,x}^2,
\end{equation}
so that (after marginalizing over the latent true $x_\star$)
\begin{equation}
\hat x_\star \mid y_{{\rm TF},\star},\theta \sim
\mathcal N\!\left(\frac{y_{{\rm TF},\star}-c}{s},\ \sigma_{1,\star}\right),
\qquad
y_\star \mid y_{{\rm TF},\star},\theta \sim \mathcal N\!\left(y_{{\rm TF},\star},\ \sigma_{\rm int,y}\right).
\end{equation}
If we take $p(y_{{\rm TF},\star})\propto 1$ on $\mathbb{R}$, then the conditional distribution of the latent
on-relation magnitude is
\begin{equation}
y_{{\rm TF},\star}\mid \hat x_\star,\theta \sim
\mathcal N\!\left(c+s\,\hat x_\star,\ |s|\,\sigma_{1,\star}\right).
\end{equation}
Marginalizing over $y_{{\rm TF},\star}$ yields a simple Gaussian predictive distribution for the galaxy's true
absolute magnitude:
\begin{equation}
y_\star \mid \hat x_\star,\sigma_{x,\star},\theta \sim
\mathcal N\!\left(
c+s\,\hat x_\star,\ \sigma_{y\mid x,\star}
\right),
\qquad
\sigma_{y\mid x,\star}^2 \equiv s^2\,\sigma_{1,\star}^2 + \sigma_{\rm int,y}^2.
\label{eq:y_given_xhat}
\end{equation}
Thus, for a fixed parameter vector $\theta$, the TF-based point prediction and uncertainty are
\begin{equation}
\mathbb E[y_\star \mid \hat x_\star,\theta] = c+s\,\hat x_\star,
\qquad
{\rm Var}(y_\star \mid \hat x_\star,\theta)= s^2(\sigma_{x,\star}^2+\sigma_{\rm int,x}^2)+\sigma_{\rm int,y}^2.
\end{equation}

\paragraph{Including a measurement model for a future magnitude observation.}
If instead we want the predictive distribution for a \emph{measured} absolute magnitude $\hat y_\star$ with known
measurement uncertainty $\sigma_{y,\star}$, then
\begin{equation}
\hat y_\star \mid \hat x_\star,\sigma_{x,\star},\theta \sim
\mathcal N\!\left(
c+s\,\hat x_\star,\ \sqrt{\sigma_{y\mid x,\star}^2+\sigma_{y,\star}^2}
\right).
\end{equation}

\subsection{Using a bounded top-hat prior on $y_{\rm TF}$}
If, as in \S\ref{sec:simple_noPV}, we take
$y_{{\rm TF},\star}\sim{\rm Uniform}(y_{\min},y_{\max})$, then
\begin{equation}
y_{{\rm TF},\star}\mid \hat x_\star,\theta
\ \propto\
\mathcal N\!\left(\hat x_\star;\frac{y_{{\rm TF},\star}-c}{s},\sigma_{1,\star}\right)\,
\mathbf{1}\{y_{\min}\le y_{{\rm TF},\star}\le y_{\max}\},
\end{equation}
i.e. a truncated normal distribution with (untruncated) location $c+s\hat x_\star$ and scale $|s|\sigma_{1,\star}$.
The resulting $p(y_\star\mid \hat x_\star,\theta)$ is no longer exactly Gaussian because it is the convolution of a
truncated normal ($y_{\rm TF}$) with a normal ($y_\star\mid y_{\rm TF}$).

A convenient and exact approach is Monte Carlo composition, conditional on each posterior draw $\theta^{(m)}$:
\begin{align}
y_{{\rm TF},\star}^{(m,k)} &\sim {\rm TruncNormal}\!\left(c^{(m)}+s^{(m)}\hat x_\star,\ |s^{(m)}|\sigma_{1,\star}^{(m)};\ [y_{\min},y_{\max}]\right),\\
y_{\star}^{(m,k)} &\sim \mathcal N\!\left(y_{{\rm TF},\star}^{(m,k)},\ \sigma_{\rm int,y}^{(m)}\right),
\end{align}
and then pool $\{y_{\star}^{(m,k)}\}$ over $(m,k)$ to obtain posterior means and credible intervals.

\subsection{Recommended reporting}
Given posterior draws $\theta^{(m)}$, a simple summary for the galaxy is:
\begin{itemize}
\item A posterior predictive median (or mean) for $y_\star$;
\item A central credible interval (e.g.\ 68\% or 95\%) from the mixture distribution in Eq.~\eqref{eq:posterior_predictive_y};
\item Optionally, a decomposition of the conditional variance in Eq.~\eqref{eq:y_given_xhat} into the contributions from
measurement uncertainty in $\hat x_\star$, intrinsic scatter in $x$, and intrinsic scatter in $y$.
\end{itemize}
This procedure propagates uncertainty in the TF calibration $(s,c)$ and in the intrinsic-scatter parameters into the
inferred magnitude for the individual galaxy.