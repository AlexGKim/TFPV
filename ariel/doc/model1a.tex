\section{Simple Model With No Peculiar Velocities}
Consider a simple model in which peculiar velocities are not included and the background cosmology is known.
In this case observed magnitudes can be converted to absolute magnitudes, and observed redshifts can be
converted to distances.

\subsection{Notation}
Indexing:
\begin{itemize}
    \item Galaxy index: $i = 1,\ldots,N_{\text{total}}$
    \item Redshift-bin index: $j = 1,\ldots,N_{\text{bins}}$
\end{itemize}

\subsection{Data}
\begin{description}
    \item[$N_{\text{bins}}$:] Number of redshift bins.
    \item[$N_{\text{total}}$:] Total number of galaxies across all bins.
    \item[$\hat x_i$:] Observed $\log\!\left(\frac{V_{\text{rot}}}{V_0}\right)$ for galaxy $i$.
    \item[$\sigma_{x,i}$:] Uncertainty on $\hat x_i$ (set to zero if not available).
    \item[$\hat y_i$:] Observed absolute magnitude for galaxy $i$.
    \item[$\sigma_{y,i}$:] Uncertainty on $\hat y_i$ (set to zero if not available).
    \item[$b_i$:] Redshift-bin index for galaxy $i$, with $b_i\in\{1,\ldots,N_{\text{bins}}\}$.
\end{description}

\subsection{Parameters}
\begin{description}
    \item[$s$:] Common slope across all redshift bins (real; assume $s\neq 0$).
    \item[$c_j$:] Intercept for redshift bin $j$, $j=1,\ldots,N_{\text{bins}}$.
    \item[$\sigma_{\text{int},x}$:] Intrinsic scatter in the $x$-direction (common to all galaxies; $\ge 0$).
    \item[$\sigma_{\text{int},y}$:] Intrinsic scatter in the $y$-direction (common to all galaxies; $\ge 0$).
    \item[$y_{\text{TF},i}$:] Latent ``on-relation'' absolute magnitude for galaxy $i$.
    \item[$x_i$:] Latent true $x$ for galaxy $i$.
    \item[$y_i$:] Latent true $y$ for galaxy $i$.
\end{description}

\subsection{Model}
\label{sec:model}
For each galaxy $i$:
\begin{align}
\hat x_i \mid x_i &\sim \mathcal N(x_i,\sigma_{x,i}),\\
\hat y_i \mid y_i &\sim \mathcal N(y_i,\sigma_{y,i}),\\
x_i \mid y_{{\rm TF},i}, s, c_{b_i} &\sim \mathcal N\!\left(\frac{y_{{\rm TF},i}-c_{b_i}}{s},\sigma_{{\rm int},x}\right),\\
y_i \mid y_{{\rm TF},i} &\sim \mathcal N(y_{{\rm TF},i},\sigma_{{\rm int},y}).
\end{align}
Measurement errors are assumed independent, and (for now) intrinsic scatters in $x$ and $y$ are independent.

\subsection{No Sample Selection}
\subsubsection{Total Likelihood}
Assuming galaxies are independent, the total likelihood is
\begin{align}
\mathcal L(s,\{c_j\},\sigma_{\text{int},x},\sigma_{\text{int},y};\{\hat x_i\},\{\hat y_i\})
= \prod_{i=1}^{N_{\text{total}}} \mathcal L_i(s,c_{b_i},\sigma_{\text{int},x},\sigma_{\text{int},y};\hat x_i,\hat y_i).
\end{align}

\subsubsection{Likelihood for an Individual Galaxy}
Define
\begin{align}
\mathcal L_i
&= p(\hat x_i,\hat y_i \mid s,c_{b_i},\sigma_{\text{int},x},\sigma_{\text{int},y}).
\end{align}
Marginalizing over latent variables:
\begin{align}
\mathcal L_i
&= \int dx_i\,dy_i\,dy_{\text{TF},i}\;
p(\hat x_i\mid x_i)\,p(\hat y_i\mid y_i)\,
p(x_i\mid y_{\text{TF},i},s,c_{b_i})\,p(y_i\mid y_{\text{TF},i})\,p(y_{\text{TF},i}).
\end{align}

Convolving intrinsic and measurement scatter gives
\begin{align}
\sigma_{1,i}^2 &= \sigma_{x,i}^2+\sigma_{\text{int},x}^2,\\
\sigma_{2,i}^2 &= \sigma_{y,i}^2+\sigma_{\text{int},y}^2,
\end{align}
so that
\begin{align}
\hat x_i \mid y_{\text{TF},i} &\sim \mathcal N\!\left(\frac{y_{\text{TF},i}-c_{b_i}}{s},\sigma_{1,i}\right),\\
\hat y_i \mid y_{\text{TF},i} &\sim \mathcal N(y_{\text{TF},i},\sigma_{2,i}),
\end{align}
and therefore the per-galaxy likelihood can be written as
\begin{align}
\mathcal L_i
&= \int dy_{\text{TF},i}\;
\mathcal N\!\left(\hat x_i;\frac{y_{\text{TF},i}-c_{b_i}}{s},\sigma_{1,i}\right)\,
\mathcal N(\hat y_i;y_{\text{TF},i},\sigma_{2,i})\,
p(y_{\text{TF},i}).
\end{align}

\subsubsection{Closed Form for a Top-Hat Prior on $y_{\text{TF},i}$}
Let $p(y_{\text{TF},i})$ be uniform on $[y_{\min},y_{\max}]$:
\begin{align}
p(y_{\text{TF},i})=
\begin{cases}
\frac{1}{y_{\max}-y_{\min}}, & y_{\min}\le y_{\text{TF},i}\le y_{\max},\\
0, & \text{otherwise}.
\end{cases}
\end{align}
Then
\begin{align}
\mathcal L_i
&= \frac{1}{y_{\max}-y_{\min}}
\int_{y_{\min}}^{y_{\max}} dy_{\text{TF},i}\;
\mathcal N\!\left(\hat x_i;\frac{y_{\text{TF},i}-c_{b_i}}{s},\sigma_{1,i}\right)\,
\mathcal N(\hat y_i;y_{\text{TF},i},\sigma_{2,i})\\
&= \frac{|s|}{y_{\max}-y_{\min}}\;
\mathcal N\!\left(\hat y_i; c_{b_i}+s\hat x_i,\ \sigma_{\text{tot},i}\right)
\left[
\Phi\!\left(\frac{y_{\max}-\mu_{*,i}}{\sigma_{*,i}}\right)
-
\Phi\!\left(\frac{y_{\min}-\mu_{*,i}}{\sigma_{*,i}}\right)
\right],
\end{align}
where $\Phi$ is the standard normal CDF and
\begin{align}
\sigma_{\text{tot},i}^2 &= s^2\sigma_{1,i}^2+\sigma_{2,i}^2,\\
\mu_{*,i} &= \frac{(s\hat x_i+c_{b_i})\,\sigma_{2,i}^2+\hat y_i\,s^2\sigma_{1,i}^2}{\sigma_{\text{tot},i}^2},\\
\sigma_{*,i}^2 &= \frac{s^2\sigma_{1,i}^2\sigma_{2,i}^2}{\sigma_{\text{tot},i}^2}.
\end{align}

\paragraph{Case of no limits on $y_{\text{TF},i}$}
If $p(y_{\text{TF},i})$ is taken to be (improper) uniform on $\mathbb R$, the truncated-CDF factor tends to 1 and
\begin{align}
\mathcal L_i \propto |s|\;\mathcal N\!\left(\hat y_i; c_{b_i}+s\hat x_i,\ \sigma_{\text{tot},i}\right).
\end{align}

\subsection{Sample Selection}
\subsubsection{Sample Selection}
In practice, sample selection can bias the analysis because the probability that an object enters the final
sample need not be independent of its measured properties.
A classic example is Malmquist bias: in a magnitude-limited survey, objects near the detection threshold are
preferentially scattered into the sample if they are intrinsically brighter (or scattered brighter) than average.

In our survey, the initial parent sample is defined by the SGA catalog.
We assume that SGA inclusion is complete above its angular-size cutoff and that, conditional on the TF model,
the SGA selection is independent of the TF observables.  Concretely, letting $S_{\rm SGA}\in\{0,1\}$ denote
SGA inclusion, we assume
\begin{align}
p(\hat x_i,\hat y_i \mid \theta, S_{{\rm SGA},i}=1) = p(\hat x_i,\hat y_i \mid \theta),
\end{align}
i.e.\ the angular-size selection does not further truncate the distribution of $(\hat x,\hat y)$ relevant for
inference on $\theta$.  Under this conditional-independence assumption, the SGA cutoff contributes only an
overall constant to the likelihood and therefore does not enter the likelihood for the TF parameters.

The observed TF sample is then sculpted by DESI fiber allocation and by redshift-measurement success.
For the moment, we assume these processes are independent of galaxy properties at fixed redshift bin, so they
enter only through an overall completeness factor and do not need to be modeled explicitly in the likelihood.
This assumption should be tested, since fiber allocation and redshift success can depend on quantities such as
apparent magnitude and surface brightness.

Finally, we apply additional quality cuts to mitigate outliers and to restrict the sample to galaxies for which
the TF model is expected to be valid (e.g.\ excluding dwarf galaxies whose dynamics are not well described by
the same TF relation).
These cuts are modeled as hard selection on the measured observables for each galaxy:
$(\hat x_i,\hat y_i)\in\mathcal S_i$, where $\mathcal S_i$ is the (galaxy-dependent) selection region.
For galaxy $i$  we take $\mathcal S_i $ to be a rectangle defined by
\begin{align}
\hat x_{\min,i}\le \hat x_i\le \hat x_{\max,i},
\qquad
\hat y_{\min,i}\le \hat y_i\le \hat y_{\max,i}.
\end{align}

\subsubsection{Model}
Given the model in \S\ref{sec:model}, one may also introduce a model for the abundance of the parent galaxy population
as a function of redshift (e.g., a constant comoving number density).
This allows the selection fraction to be computed from the selection function, and hence yields a derived prediction for the expected number of observed galaxies, $\bar N$. Because $\bar N$ depends on the TF parameters through the selection probability, including an 
count term to the likelihood provides additional constraining power on those parameters.

For simplicity and tractability, we do not introduce a model for the parent population abundance at this stage,
and instead treat the expected number of observed galaxies $\bar{N}$ as a free parameter in the likelihood. 
This allows us to constrain the TF parameters from the observed distribution of $(\hat x,\hat y)$,
while avoiding the need to specify a model for the parent population abundance and its evolution with redshift.

\subsubsection{Total Likelihood}
Let $\theta \equiv \big(s,\{c_j\},\sigma_{\text{int},x},\sigma_{\text{int},y}\big)$ and let $N$ denote the expected
number of galaxies in the \emph{observed} (selected) sample. Let $N_{\text{total}}$ be the observed number of galaxies.

Assuming a Poisson model for the total number of selected galaxies and conditional independence of galaxy measurements,
the total likelihood is
\begin{align}
\mathcal L(\theta, N;\{\hat x_i\},\{\hat y_i\})
&=
\frac{e^{-N}N^{N_{\text{total}}}}{N_{\text{total}}!}
\ \prod_{i=1}^{N_{\text{total}}}
p(\hat x_i,\hat y_i \mid \theta, S_i=1),
\end{align}
where $S_i=1$ denotes that galaxy $i$ is included by the selection.

Writing the selected-sample (truncated) per-galaxy density explicitly,
\begin{align}
p(\hat x_i,\hat y_i \mid \theta, S_i=1, b_i=j)
&=
\frac{
p(\hat x_i,\hat y_i \mid \theta, b_i=j)\,
\mathbf 1\!\left[(\hat x_i,\hat y_i)\in\mathcal S_j\right]
}{
P(S=1\mid \theta, b_i=j)
},
\\
P(S=1\mid \theta, b_i=j)
&=
\int\!\!\!\int_{(\hat x,\hat y)\in\mathcal S_j}
p(\hat x,\hat y \mid \theta, b_i=j)\, d\hat x\, d\hat y,
\end{align}
and the selection region for bin $j$ can be taken as the rectangle
\begin{align}
\mathcal S_j
=
\Big\{(\hat x,\hat y):
\hat x_{\min,j}\le \hat x\le \hat x_{\max,j},\ 
\hat y_{\min,j}\le \hat y\le \hat y_{\max,j}
\Big\}.
\end{align}

\subsection{Priors}
Use flat priors for $(s,\{c_j\})$ and weakly informative priors for the intrinsic scatters:
\begin{align}
\sigma_{\text{int},x} &\sim \text{Half-Cauchy}\!\left(0,\ 5\cdot \mathrm{sd}[\{\hat x_i\}]\right),\\
\sigma_{\text{int},y} &\sim \text{Half-Cauchy}\!\left(0,\ 5\cdot \mathrm{sd}[\{\hat y_i\}]\right).
\end{align}